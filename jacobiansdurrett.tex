\documentclass[12pt]{article}
\usepackage[utf8]{inputenc}
\usepackage[margin=2cm]{geometry}
\usepackage{fullpage,setspace,amssymb,mathrsfs,amsmath,amsthm,xcolor,cancel,gensymb,hyperref,graphicx,mathtools}
\usepackage{indentfirst}
\usepackage[shortlabels]{enumitem}
\usepackage{tikz}
\usepackage{amsfonts}
\setlength{\parskip}{1em}
\newenvironment{solution}{\renewcommand{\proofname}{Solution}\begin{proof}}{\end{proof}}

\title{Misc. Durrett Problems}
\author{Jacobian}
\date{October 2023}

\begin{document}

\maketitle

\textbf{1.3.2. (NPR)} We will prove that when $X_1, X_2,..., X_n$ are random variables, then it is also true that $X_1+X_2+...+X_n$ is a random variable. To do so, it is enough to verify that $X_1+X_2$ is a random variable, and the general case will follow by induction.

By \textbf{Theorem 1.3.1}, it is enough to show that $(X_1+X_2)^{-1}((-\infty, a))\in \mathcal{F}$ for any $a\in \mathbb{Q}$, since we have seen that these sets generate the $\sigma$-algebra $\mathcal{R}.$

We claim 
\begin{equation}
    (X_1+X_2)^{-1}((-\infty,a)) = \bigcup_{p\in\mathbb{Q}} \left[X_2^{-1}((-\infty,p))\,\cap\,X_1^{-1}((-\infty,a-p))\right].\label{eq:1}
\end{equation}
\indent Indeed, one direction is immediate. If $X_2(\omega)<p$ for some $p\in\mathbb{Q}$ and $X_1(\omega)<a-p,$ then $X_1+X_2<a.$

Conversely, if $X_1(\omega)+X_2(\omega)<a,$ then by the density of the rational numbers in $\mathbb{R},$ we may pick $q\in\mathbb{Q}$ between $X_1(\omega)+X_2(\omega)$ and $a.$ Since $a-q>0$ by construction, we may again use the density of $\mathbb{Q}$ to pick $p\in\mathbb{Q}$ such that $X_2(\omega)<p<X_2(\omega) + (a-q).$ Rearranging the right-side inequality yields $p+q-a<X_2(\omega).$ Hence,
$$X_1(\omega) < q-X_2(\omega) < q-(p+q-a) = a-p,$$
as desired. Recall that $X_2(\omega)<p$ by construction. Hence, the sets are equal, and the right-hand side of \eqref{eq:1} is a countable union of intersections of sets that are in $\mathcal{F},$ as $X_1$ and $X_2$ are assumed to be random variables. Hence, $(X_1+X_2)^{-1}((-\infty,a))\in \mathcal{F}$ by the axioms of a $\sigma$-algebra.

\end{document}
